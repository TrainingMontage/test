\documentclass{scrreprt}
\usepackage[shellescape]{gmp}
\usepackage{enumitem}
\begin{document}

    \chapter{Wayside Controller}
    \section{Use Cases}

    \begin{center}
        \resizebox{!}{.8\textheight}{
            \begin{mpost}
                input metauml;
                vardef drawComponentVisualStereotype(text ne)= relax enddef;
                Actor.ctc("CTC");
                Actor.weng("Wayside", "Engineer");
                Actor.simrunner("Simulation", "Runner");
                %======== CTC Uses =========
                Usecase.uGetOcc      ("Get Occupancy");
                Usecase.uGetStatus   ("Get Block Status");
                Usecase.uGetSig      ("Get Signal");
                Usecase.uGetCross    ("Get Crossing");
                Usecase.uGetSwitch   ("Get Switch State");
                Usecase.uSetStatus   ("Set Block Status");
                Usecase.uToggleSwitch("Toggle Switch");
                Usecase.uSuggest     ("Suggest Speed", "& Authority");
                %======== Wayside Engineer Uses =========
                Usecase.uPLC("Upload PLC");
                %======== SimRunner Uses =========
                Usecase.uManSuggest("Manual Suggestion");
                Usecase.uViewData("View Track State");
                Usecase.uMultWindow("Open Multiple", "Windows");
                %======== This Component? =========
                Component.Wayside("WaysideController")(
                    uGetOcc,
                    uGetStatus,
                    uGetSig,
                    uGetCross,
                    uGetSwitch,
                    uSetStatus,
                    uToggleSwitch,
                    uSuggest,
                    uPLC,
                    uManSuggest,
                    uViewData,
                    uMultWindow
                );
                %======== CTC use cases ========
                topToBottom.left(10)(
                    uGetOcc,
                    uGetStatus,
                    uGetSig,
                    uGetCross,
                    uGetSwitch,
                    uSetStatus,
                    uToggleSwitch,
                    uSuggest
                );
                leftToRight(25)(ctc, uGetCross);
                %======== WaysideEngineer/SimRunner use cases ========
                topToBottom.right(10)(
                    uManSuggest,
                    uViewData,
                    uMultWindow
                );
                leftToRight(70)(uGetOcc, uPLC);
                leftToRight(25)(uPLC, weng);
                leftToRight(25)(uViewData, simrunner);
                topToBottom(100)(weng, simrunner);
                %======== Draw that shit!! ========
                drawObjects(
                    ctc,weng,simrunner,TrackModel,
                    uGetOcc,
                    uGetStatus,
                    uGetSig,
                    uGetCross,
                    uGetSwitch,
                    uSetStatus,
                    uToggleSwitch,
                    uSuggest,
                    uPLC,
                    uManSuggest,
                    uViewData,
                    uMultWindow
                );
                %======== CTC Lines ========
                link(association)(pathCut(ctc, uGetOcc)      (ctc.e -- uGetOcc.w));
                link(association)(pathCut(ctc, uGetStatus)   (ctc.e -- uGetStatus.w));
                link(association)(pathCut(ctc, uGetSig)      (ctc.e -- uGetSig.w));
                link(association)(pathCut(ctc, uGetCross)    (ctc.e -- uGetCross.w));
                link(association)(pathCut(ctc, uGetSwitch)   (ctc.e -- uGetSwitch.w));
                link(association)(pathCut(ctc, uSetStatus)   (ctc.e -- uSetStatus.w));
                link(association)(pathCut(ctc, uToggleSwitch)(ctc.e -- uToggleSwitch.w));
                link(association)(pathCut(ctc, uSuggest)     (ctc.e -- uSuggest.w));
                %======== WaysdideEng Lines ========
                link(association)(pathCut(weng, uPLC)(weng.w -- uPLC.e));
                %======== SimRunner Lines ========
                link(association)(pathCut(simrunner, uManSuggest)(simrunner.w -- uManSuggest.e));
                link(association)(pathCut(simrunner, uViewData)(simrunner.w -- uViewData.e));
                link(association)(pathCut(simrunner, uMultWindow)(simrunner.w -- uMultWindow.e));
            \end{mpost}
        }
    \end{center}

\providecommand{\use}[1]{\subsection{Use Case: #1}}

    %========= CTC User Stories ========
    \use{Get Occupancy}
    \begin{enumerate}
        \item CTC calls ``isOccupied()'' with the proper block ID.
        \item WC returns true if block is occupied, false otherwise.
    \end{enumerate}

    \use{Get Block Status}
    \begin{enumerate}
        \item CTC calls ``getBlockStatus()'' with the proper block ID.
        \item WC returns one of ``OPERATIONAL'', ``BROKEN'', ``REPAIR''.
    \end{enumerate}

    \use{Get Signal}
    \begin{enumerate}
        \item CTC calls ``getSignal()'' with the proper block ID.
        \item WC returns true if signal is active, false otherwise.
    \end{enumerate}

    \use{Get Crossing}
    \begin{enumerate}
        \item CTC calls ``getCrossing()'' with the proper block ID.
        \item WC returns true if railroad crossing is active, false otherwise.
    \end{enumerate}

    \use{Get Switch State}
    \begin{enumerate}
        \item CTC calls ``getSwtich()'' with the proper block ID.
        \item WC returns true if switch is in active position, false otherwise.
    \end{enumerate}

    \use{Set Block Status}
    \begin{enumerate}
        \item CTC calls ``setBlockStatus()'' with the proper block ID, and the desired state.
        \item WC sends signal to TrackModel to set the status of the block with the given ID to the desired state.
        \item WC then returns the actual status of the block after assignment.
    \end{enumerate}

    \use{Toggle Switch State}
    \begin{enumerate}
        \item CTC calls ``toggleSwitch()'' with the proper block ID.
        \item WC validates this is safe.
        \item WC sends signal to TrackModel to toggle specified switch.
        \item WC then returns switch position.
    \end{enumerate}
    \subsubsection{Variant: Unsafe Toggle}
    \begin{enumerate}[label=3\alph*.]
        \item If WC fails to validate that this is safe, it does not send this message to TrackModel.
    \end{enumerate}

    \use{Suggest Speed \& Authority}
    \begin{enumerate}
        \item CTC calls ``suggest()'' with suggestion encoded in the proper format.
        \item WC validates that this input is safe.
        \item WC then sends the following to TrackModel:
            \begin{itemize}
                \item The given speed and authority.
                \item The position of switches necessary for this route.
                \item Activates any crossings necessary for this route.
                \item The state of all signals necessary for this route.
            \end{itemize}
    \end{enumerate}

    \subsubsection{Variant: Unsafe Suggestion}
    \begin{enumerate}[label=\arabic*a., start=3]
        \item If WC does not find this suggestion to be safe, then it will instead send the following to the TrackModel:
            \begin{itemize}
                \item No authority to all blocks.
                \item 0 speed to all blocks.
                \item Deactivates all crossing which are not currently occupied.
                \item Deactivates all signals.
            \end{itemize}
    \end{enumerate}

    %========== Wayside Users =========
    \use{Upload PLC}
    \begin{enumerate}
        \item 
    \end{enumerate}

    \use{Manually Suggest Speed \& Authority}
    \begin{enumerate}
        \item 
    \end{enumerate}

    \use{View Track State}
    \begin{enumerate}
        \item 
    \end{enumerate}

    \use{Open Multiple Wayside Contolle Windows}
    \begin{enumerate}
        \item 
    \end{enumerate}
\end{document}