\documentclass{scrreprt}
\usepackage[shellescape]{gmp}
\usepackage{listings}
\usepackage{underscore}
\usepackage[bookmarks=true]{hyperref}
\usepackage[utf8]{inputenc}
\usepackage[english]{babel}
\usepackage{enumitem}
\usepackage{graphicx}
\usepackage{xcolor}
\usepackage{fancyhdr}
\usepackage{tikz}
\usepackage{pgf-umlsd}

%%%%%% header and footer info
\pagestyle{fancy}
\fancyhf{}
\rhead{Software Requirements Specification}
\lhead{Training Montage}
\cfoot{\thepage}

%%%%%% custom list definition
\newlist{numonly}{enumerate}{10}
\setlist[numonly]{label*=\arabic*.}

\hypersetup{
    pdftitle={Software Requirement Specification},    % title
    pdfauthor={Training Montage},                     % author
    pdfsubject={TeX and LaTeX},                        % subject of the document
    pdfkeywords={TeX, LaTeX, graphics, images}, % list of keywords
    colorlinks=true,       % false: boxed links; true: colored links
    linkcolor=blue,       % color of internal links
    citecolor=black,       % color of links to bibliography
    filecolor=black,        % color of file links
    urlcolor=purple,        % color of external links
    linktoc=page            % only page is linked
}

\begin{document}

\begin{flushright}
    \rule{16cm}{5pt}\vskip1cm
    \begin{bfseries}
        \Huge{TRACK MODEL USE CASE, CLASS, AND SEQUENCE DIAGRAMS}\\
        \vspace{.9cm}
        for\\
        \vspace{.9cm}
        COE 1186 Project\\
        \vspace{.9cm}
        % \LARGE{Version \myversion approved}\\
        \vspace{.9cm}
        Prepared by:\\
        Alec Rosenbaum\\
        \vspace{4.9cm}
        Training Montage\\
        \vspace{.9cm}
        \today\\
    \end{bfseries}
\end{flushright}

\tableofcontents

\chapter{Track Model}

\section{Use Cases}

\begin{center}
\resizebox{!}{.8\textheight}{
	\begin{mpost}
		input metauml;
		vardef drawComponentVisualStereotype(text ne)= relax enddef;
		Actor.ctc("CTC");
		Actor.wayside("Wayside");
		Actor.trainmodel("Train Model");
		Actor.traincontroller("Train Controller");
	    % 
	    Usecase.uA("Get Static Block");
	    Usecase.uB("Get Static Switch");
	    %
	    Usecase.uC("Get Occupancy");
		Usecase.uD("Set Suggested", "Speed");
		Usecase.uDD("Set Suggested", "Authority");
		Usecase.uE("Set Maintainance");
		Usecase.uF("Set Switch State");
		Usecase.uG("Set Signals");
		Usecase.uH("Set Railroad", "Crossing");
		Usecase.uI("Get Block State");
		Usecase.uJ("Get Signal");
		Usecase.uK("Get Crossing");
		%
		Usecase.uL("Initalize Train");
		Usecase.uM("Get Suggested", "Authority");
		Usecase.uN("Get Suggested", "Speed");
		Usecase.uO("Get Passengers");
		Usecase.uP("Get Grade");
		Usecase.uQ("Get If Track", "Is Icy");
		Usecase.uR("Get Beacon Data");
		Usecase.uS("Teardown Train");
	    %
	    Component.TrackModel("Track Model")(uA,uB,uC,uD,uDD,uE,uF,uG,uH,uI,uJ,uK,uL,uM,uN,uO,uP,uQ,uR,uS);
	    % 
		topToBottom.left(10)(uA,uB,uC,uD,uDD,uE,uF,uG,uH,uI,uJ,uK);
		leftToRight(50)(ctc, uB);
		leftToRight(50)(traincontroller, uD);
		leftToRight(50)(wayside, uG);
		%
		topToBottom.right(10)(uL,uM,uN,uO,uP,uQ,uR,uS);
		leftToRight(100)(uC, uL);
		leftToRight(50)(uL, trainmodel);
	    % 
	    drawObjects(ctc,wayside,trainmodel,traincontroller,TrackModel,uA,uB,uC,uD,uDD,uE,uF,uG,uH,uI,uJ,uK,uL,uM,uN,uO,uP,uQ,uR,uS);
		%
	    link(association)(pathCut(ctc, uA)(ctc.e -- uA.w));
	    link(association)(pathCut(ctc, uA)(ctc.e -- uB.w));
	    link(association)(pathCut(traincontroller, uA)(traincontroller.e -- uA.w));
	    link(association)(pathCut(traincontroller, uA)(traincontroller.e -- uB.w));
	    link(association)(pathCut(wayside, uA)(wayside.e -- uA.w));
	    link(association)(pathCut(wayside, uB)(wayside.e -- uB.w));
	    link(association)(pathCut(wayside, uC)(wayside.e -- uC.w));
	    link(association)(pathCut(wayside, uD)(wayside.e -- uD.w));
	    link(association)(pathCut(wayside, uE)(wayside.e -- uE.w));
	    link(association)(pathCut(wayside, uF)(wayside.e -- uF.w));
	    link(association)(pathCut(wayside, uG)(wayside.e -- uG.w));
	    link(association)(pathCut(wayside, uH)(wayside.e -- uH.w));
	    link(association)(pathCut(wayside, uI)(wayside.e -- uI.w));
	    link(association)(pathCut(wayside, uJ)(wayside.e -- uJ.w));
	    link(association)(pathCut(wayside, uK)(wayside.e -- uK.w));
	    %
	    link(association)(pathCut(trainmodel, uA)(trainmodel.w -- uA.e));
	    link(association)(pathCut(trainmodel, uB)(trainmodel.w -- uB.e));
	    link(association)(pathCut(trainmodel, uL)(trainmodel.w -- uL.e));
	    link(association)(pathCut(trainmodel, uM)(trainmodel.w -- uM.e));
	    link(association)(pathCut(trainmodel, uN)(trainmodel.w -- uN.e));
	    link(association)(pathCut(trainmodel, uO)(trainmodel.w -- uO.e));
	    link(association)(pathCut(trainmodel, uP)(trainmodel.w -- uP.e));
	    link(association)(pathCut(trainmodel, uQ)(trainmodel.w -- uQ.e));
	    link(association)(pathCut(trainmodel, uR)(trainmodel.w -- uR.e));
	    link(association)(pathCut(trainmodel, uS)(trainmodel.w -- uS.e));
	\end{mpost}
}
\end{center}

\subsection{Use Case: Get Static Block}
\begin{enumerate}
	\item User calls getStaticBlock(blockId)
	\item Intstance of StaticBlock populated with static data is returned.
\end{enumerate}

\subsection{Use Case: Get Static Switch}
\begin{enumerate}
	\item User calls getStaticSwitch(switchId)
	\item Intstance of StaticSwitch populated with static data is returned.
\end{enumerate}

\subsection{Use Case: Get Occupancy}
\begin{enumerate}
	\item Wayside Actor calls isOccupied(blockId)
	\item True is returned if block is occupied, False if block is unoccupied.
\end{enumerate}

\subsection{Use Case: Set Suggested Speed}
\begin{enumerate}
	\item Wayside calls setSpeed(blockId, speed)
	\item Suggested speed is set for that block.
\end{enumerate}

\subsection{Use Case: Set Suggested Speed}
\begin{enumerate}
	\item Wayside calls setAuthority(blockId, authority)
	\item Authority is set for that block.
\end{enumerate}

\subsection{Use Case: Set Maintainance}
\begin{enumerate}
	\item Wayside calls setStatus(blockId, BlockStatus.MAINTAIN)
	\item The block status of blockId is set to be Maintainance.
\end{enumerate}


\section{Class Diagram}

\begin{center}
\resizebox{!}{.9\textheight}{
	\begin{mpost}
		input metauml;
	    % Track Model
	    Class.TrackModel("TrackModel")()(
	    	"int getTrainAuthority(int trainId)",
			"double getTrainSpeed(int trainId)",
			"boolean setSignal(int blockId, boolean value)",
			"boolean getSignal(int blockId)",
			"boolean setSwitch(int blockId, boolean value)",
			"boolean getSwitch(int blockId)",
			"boolean setAuthority(int blockId, boolean authority)",
			"int setSpeed(int blockId, int speed)",
			"boolean setCrossing(int blockId, boolean active)",
			"boolean getCrossing(int blockId)",
			"byte[] getTrainBeacon(int trainId)",
			"int getPassengers(int trainId)",
			"boolean isIcyTrack(int trainId)",
			"StaticBlock getStaticBlock(int blockId)",
			"double getGrade(int trainId)",
			"boolean isOccupied(int blockId)",
			"BlockStatus getStatus(int blockId)"
		);
		% Database
	    Class.Database("Database")()(
	    	"ResultSet execute(Statement)",
		);
		% Static Block
		Class.StaticBlock("StaticBlock")()(
			"-int setId(int id)",
			"int getId()",
			"-String setRegion(String region)",
			"String getRegion()",
			"-double setGrade(double grade)",
			"double getGrade()",
			"-double setElevation(double elevation)",
			"double getElevation()",
			"-double setLength(double length)",
			"double getLength()",
			"-String setStation(String station)",
			"String getStation()",
			"-StaticSwitch setStaticSwitch(StaticSwitch staticSwitch)"
		);
	    % StaticSwitch
	    Class.StaticSwitch("StaticSwitch")()(
			"StaticBlock getRoot()",
			"-StaticBlock setRoot(StaticBlock root)",
			"StaticBlock getInactiveLeaf()",
			"-StaticBlock setInactiveLeaf(StaticBlock inactiveLeaf)",
			"StaticBlock getActiveLeaf()",
			"-StaticBlock setActiveLeaf(StaticBlock activeLeaf)"
	    );
		% 	
		topToBottom(30)(TrackModel, Database, StaticBlock, StaticSwitch);
		% 
	    drawObjects(TrackModel, Database, StaticBlock, StaticSwitch);
		%
		% StaticSwitch -> StaticBlock 1..3 
	    link(compositionUni)(pathStepX(StaticBlock.e, StaticSwitch.e, 30));
	    item(iAssoc)("1")(obj.nw = StaticSwitch.e + (15,0));
	    item(iAssoc)("3")(obj.nw = StaticBlock.e + (5,-5));
	    % TrackModel -> StaticSwitch 1..*
	    link(compositionUni)(pathStepX(StaticSwitch.e + (0,-20), TrackModel.e, 55));
	    item(iAssoc)("*")(obj.nw = StaticSwitch.e + (15,-20));
	    item(iAssoc)("1")(obj.nw = TrackModel.e + (5,-5));
	    % TrackModel -> StaticBlock 1..*
	    link(compositionUni)(pathStepX(StaticBlock.w, TrackModel.w, -25));
	    item(iAssoc)("*")(obj.nw = StaticBlock.w + (-20,0));
	    item(iAssoc)("1")(obj.nw = TrackModel.w + (-20,0));
	    % TrackModel -> Database 1..1
	    link(associationUni)(pathStepX(Database.n, TrackModel.s, 0));
	    item(iAssoc)("1")(obj.ne = TrackModel.s + (-5,-10));
	    item(iAssoc)("1")(obj.ne = Database.n + (0,10));
	    % StaticBlock -> StaticSwitch 1..0-1
	    link(associationUni)(pathStepX(StaticBlock.w + (0, -20), StaticSwitch.w, -25));
	    item(iAssoc)("1")(obj.nw = StaticBlock.w + (-20,-20));
	    item(iAssoc)("0..1")(obj.nw = StaticSwitch.w + (-20,-5));
	\end{mpost}
}
\end{center}

\section{Sequence Diagrams}

\subsection{Call to TrackModel.getStaticBlock(id)}

\begin{center}
\resizebox{\textwidth}{!}{
 	\begin{sequencediagram}
	\newthread{sim}{Caller}
	\newinst[4]{tm}{TrackModel}
	\newinst[2]{sb}{StaticBlock}
	\newinst{ss}{StaticSwitch}
	
	\begin{sdblock}{Initialization}{Before Using the API}
		\begin{call}{sim}{Init()}{tm}{}
		\end{call}
	\end{sdblock}
	\begin{call}{sim}{TrackModel.getStaticBlock(id)}{tm}{}
		\begin{call}{tm}{StaticBlock()}{sb}{}
		\end{call}
		\begin{sdblock}{If A Switch is Attached}{}
			\begin{call}{tm}{StaticSwitch()}{ss}{}
			\end{call}
		\end{sdblock}
	\end{call}
	\end{sequencediagram}
}
\end{center}

\subsection{Call to TrackModel.getStaticSwitch(id)}

\begin{center}
\resizebox{\textwidth}{!}{
 	\begin{sequencediagram}
	\newthread{sim}{Caller}
	\newinst[5]{tm}{TrackModel}
	\newinst[2]{ss}{StaticSwitch}
	
	\begin{sdblock}{Initialization}{Before Using the API}
		\begin{call}{sim}{Init()}{tm}{}
		\end{call}
	\end{sdblock}
	\begin{call}{sim}{TrackModel.getStaticSwitch(id)}{tm}{}
		\begin{call}{tm}{StaticSwitch()}{ss}{}
		\end{call}
		\begin{sdblock}{For Root, Default, Active}{}
			\begin{call}{ss}{getStaticBlock(id)}{tm}{}
			\end{call}
		\end{sdblock}
	\end{call}
	\end{sequencediagram}
}
\end{center}

\end{document}