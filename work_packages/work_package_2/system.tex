\documentclass{scrreprt}
\usepackage[shellescape]{gmp}
\usepackage{listings}
\usepackage{underscore}
\usepackage[bookmarks=true]{hyperref}
\usepackage[utf8]{inputenc}
\usepackage[english]{babel}
\usepackage{enumitem}
\usepackage{graphicx}
\usepackage{xcolor}
\usepackage{fancyhdr}
\usepackage{tikz}
\usepackage{pgf-umlsd}

%%%%%% header and footer info
\pagestyle{fancy}
\fancyhf{}
\rhead{System Software Design Description}
\lhead{Training Montage}
\cfoot{\thepage}

%%%%%% custom list definition
\newlist{numonly}{enumerate}{10}
\setlist[numonly]{label*=\arabic*.}

\hypersetup{
    pdftitle={System Software Design Description},    % title
    pdfauthor={Training Montage},                     % author
    pdfsubject={TeX and LaTeX},                        % subject of the document
    pdfkeywords={TeX, LaTeX, graphics, images}, % list of keywords
    colorlinks=true,       % false: boxed links; true: colored links
    linkcolor=blue,       % color of internal links
    citecolor=black,       % color of links to bibliography
    filecolor=black,        % color of file links
    urlcolor=purple,        % color of external links
    linktoc=page            % only page is linked
}

\begin{document}

\begin{flushright}
    \rule{16cm}{5pt}\vskip1cm
    \begin{bfseries}
        \Huge{SYSTEM SOFTWARE DESIGN DESCRIPTION}\\
        \vspace{.9cm}
        for\\
        \vspace{.9cm}
        COE 1186 Project\\
        \vspace{.9cm}
        % \LARGE{Version \myversion approved}\\
        \vspace{.9cm}
        Prepared by:\\
        Alec Rosenbaum\\
        Aric Hudson\\
        Isaac Goss\\
        Mitch Moran\\
	    Parth Dadhania\\
        \vspace{1.9cm}
        Training Montage\\
        \vspace{.9cm}
        \today\\
    \end{bfseries}
\end{flushright}

\tableofcontents

\chapter{System Architecture}

\section{Use Cases}

\begin{center}
\resizebox{!}{.7\textheight}{
	\begin{mpost}
	    input metauml;
	    vardef drawComponentVisualStereotype(text ne)= relax enddef;
	    Actor.dispatch("Dispatcher");
	    Actor.weng("Wayside Engineer");
	    Actor.teng("Train Engineer");
	    Actor.driver("Train Driver");
	    Actor.pass("Passengers");
	    % 
	    Usecase.uA("Dispatch Train");
	    Usecase.uB("Upload Schedule");
	    Usecase.uC("Close Track for", "Maintenance");
	    Usecase.uD("Toggle Switch", "for Testing");
	    Usecase.uE("Upload PLC");
	    Usecase.uF("Set Kp and Ki");
	    Usecase.uG("Set Target Velocity");
	    Usecase.uH("Apply Service", "Brake");
	    Usecase.uI("Turn Lights", "On and Off");
	    Usecase.uJ("Open and", "Close Doors");
	    Usecase.uK("Set Internal", "Temperature");
	    Usecase.uL("Apply Emergency", "Brake");
	    %
	    Component.System("System")(uA,uB,uC,uD,uE,uF,uG,uH,uI,uJ,uK,uL);
	    % 
	    topToBottom.left(10)(uE,uA,uB,uC,uD,uF);
	    topToBottom.right(10)(uG,uH,uI,uJ,uK,uL);
	    leftToRight(30)(dispatch, uB);
	    leftToRight(30)(weng, uE);
	    leftToRight(30)(teng, uF);
	    leftToRight(30)(uI, driver);
	    leftToRight(30)(uL, pass);
	    leftToRight(10)(uE, uG);
	    leftToRight(10)(uA, uH);
	    leftToRight(10)(uB, uI);
	    leftToRight(10)(uC, uJ);
	    leftToRight(10)(uD, uK);
	    leftToRight(10)(uF, uL);
	    %
	    % 
	    drawObjects(dispatch,weng,teng,driver,pass,System,uA,uB,uC,uD,uE,uF,uG,uH,uI,uJ,uK,uL);
		%
	    link(association)(pathCut(dispatch, uA)(dispatch.e -- uA.w));
	    link(association)(pathCut(dispatch, uB)(dispatch.e -- uB.w));
	    link(association)(pathCut(dispatch, uC)(dispatch.e -- uC.w));
	    link(association)(pathCut(dispatch, uD)(dispatch.e -- uD.w));
	    %
	    link(association)(pathCut(weng, uE)(weng.e -- uE.w));
	    %
	    link(association)(pathCut(teng, uF)(teng.e -- uF.w));
	    %
	    link(association)(pathCut(driver, uG)(driver.w -- uG.e));
	    link(association)(pathCut(driver, uH)(driver.w -- uH.e));
	    link(association)(pathCut(driver, uI)(driver.w -- uI.e));
	    link(association)(pathCut(driver, uJ)(driver.w -- uJ.e));
	    link(association)(pathCut(driver, uK)(driver.w -- uK.e));
	    link(association)(pathCut(driver, uL)(driver.w -- uL.e));
	    %
	    link(association)(pathCut(pass, uL)(pass.w -- uL.e));
	    %
	\end{mpost}
}
\end{center}

\subsection{Use Case: Dispatch Train}
\begin{enumerate}
	\item Dispatcher clicks the "New Train" GUI button.
	\item Dispatcher fill the fields (block number, suggested speed, suggested authority, and destination block) and presses "Dispatch Train" GUI button.
	\item CTC will call the train model to create a new train and train controller.
	\item The train register itself with the track model.
	\item The dispatcher will now see a train on the specified block on the map.
\end{enumerate}

\subsection{Variant: Invalid Inputs}
\begin{enumerate}[label=\arabic*a., start=3]
	\item If the starting block is not next to the yard or an input is invalid, no train will be created and the invalid input will be outlined in red.
\end{enumerate}

\subsection{Use Case: Upload Schedule}
\begin{enumerate}
	\item Dispatcher selects "Upload Schedule" from the File menu.
	\item Dispatcher selects a schedule file on the computer.
	\item Schedule is parsed and is displayed by train to the dispatcher.
\end{enumerate}

\subsection{Variant: Unable to Read File}
\begin{enumerate}[label=\arabic*a., start=3]
	\item If the file could not be read, a popup will report the error to the dispatcher.
\end{enumerate}

\subsection{Use Case: Close Track for Maintenance}
\begin{enumerate}
	\item Dispatcher selects a block from the map.
	\item Dispatcher clicks "Close for Maintenance" GUI button.
	\item Block state displays "Maintenance" to the dispatcher
\end{enumerate}

\subsection{Use Case: Toggle Switch for Testing}
\begin{enumerate}
	\item Dispatcher selects a block with a switch from the map.
	\item Dispatcher clicks "Toggle Switch" GUI button.
	\item If wayside deems it safe, switch state will change and be displayed to the dispatcher.
\end{enumerate}

\subsection{Use Case: Upload PLC}
\begin{enumerate}
    \item Wayside Engineer (WE) pushes ``Upload PLC'' button.
    \item A File Chooser popup appears.
    \item WE chooses desired file.
    \item Popup appears stating PLC has successfully been uploaded.
\end{enumerate}
\subsubsection{Variant: Unable to Read File}
\begin{enumerate}[label = \arabic*a., start = 4]
    \item If the selected file cannot be read for any reason, then a popup will appear reporting the error.
\end{enumerate}
\subsubsection{Variant: Invalid PLC}
\begin{enumerate}[label = \arabic*b., start = 4]
    \item If the selected file does not contain valid PLC code, then a popup will appear reporting the error.
\end{enumerate}

\subsection{Use Case: Get Power}
\begin{enumerate}
	\item If the Train needs to stop, or slow down, the Train Controller applies the brake.
	\item Train Controller calls getSuggestedSpeed from the Train Model to obtain the suggested velocity for the Train.
	\item Train Controller calls getMaxPower from the Train Model to obtain the maximum power output for the Train.
	\item Train Controller calls getVelocity from the Train Model to obtain the Train's current velocity.
	\item Train Controller calls getAuthority from the Train Model to obtain the Train's current authority.
	\item Train Controller calculates the safe speed for the train, given its authority.
	\item Train Controller returns a power command for the  train, taking into account its power limits, suggested speed, and current velocity.
\end{enumerate}

\subsection{Use Case: Set Target Velocity}
\begin{enumerate}
	\item The Driver toggles “manual mode” on the UI.
	\item The Driver manually sets the train to a desired speed.
	\item The Driver hits “Apply Changes”
	\item Future calls to Get Power create power commands to approach this manually entered speed while “manual mode” is active.
\end{enumerate}

\subsection{Use Case: Apply Service Brakes}
\begin{enumerate}
	\item The Driver toggles “manual mode” on the UI.
	\item The Driver presses the "Service Brake" button.
	\item Future calls to Get Power apply the service brake while the button is toggled.
\end{enumerate}

\subsection{Use Case: Apply Emergency Brakes}
\begin{enumerate}
	\item The Driver toggles “manual mode” on the UI.
	\item The Driver presses the "Emergency Brake" button.
	\item Future calls to Get Power apply the emergency brake while the button is toggled.
\end{enumerate}

\subsection{Use Case: Operate Lights}
\begin{enumerate}
	\item The Driver toggles “manual mode” on the UI.
	\item The Driver toggles the lights "ON" or "OFF" buttons
	\item The Train Controller switches the lights to be "on" or "off" accordingly.
\end{enumerate}

\subsection{Use Case: Operate Doors}
\begin{enumerate}
	\item The Driver toggles “manual mode” on the UI.
	\item The Driver toggles the door "OPEN" or "CLOSED" buttons
	\begin{enumerate}
		\item There are options for both the right and left doors.
	\end{enumerate}
	\item The Train Controller switches the appropriate doors to be "open" or "closed" accordingly.
\end{enumerate}

\subsection{Use Case: Set Temperature}
\begin{enumerate}
	\item The Driver toggles “manual mode” on the UI.
	\item The Driver manually sets the train's internal temperature
	\item The Driver hits “Apply Changes”
	\item The Train Controller applies the desired temperature to the train interior while manual mode is active.
\end{enumerate}

\subsection{Use Case: Initialize TrainController}
\begin{enumerate}
	\item Either the SimulationRunner or the Train Model calls the basic constructor or init() function.
	\item A train controller is initialized with an ID of 0 and using the default constructor for a Train.
	\begin{enumerate}
		\item Alternative constructors exist to set a specific train.
	\end{enumerate}
	\item The actor will then be able to set Ki and Kp, if desired.
\end{enumerate}


\section{Class Diagram}

\begin{center}
\resizebox{\textwidth}{!}{
	\begin{mpost}
		input metauml;
        % CTC
        Class.CTC("CTC")()(
            "init()",
            "..."
        );
        % Wayside
        Class.Wayside("Wayside")()(
            "init()",
            "..."
        );
	 	% Track Model
        Class.TrackModel("TrackModel")()(
            "init()",
            "..."
        );
        % Train Model
        Class.TrainModel("TrainModel")()(
            "init()",
            "..."
        );
        % Train Controller
        Class.TrainController("TrainController")()(
            "init()",
            "..."
        );
        % Environment
        Class.Environment("Environment")(
            "time",
            "temperature"
        )();
        % SimRunner
        Class.SimRunner("SimRunner")()(
            "main(args[])"
        );
        % Convert
        Class.Convert("Convert")()(
            "metersToFeet()",
            "..."
        );
        % BlockStatus
        Class.BlockStatus("BlockStatus")(
            "OPERATIONAL",
            "BROKEN",
            "..."
        )();
        % Suggestion
        Class.Suggestion("Suggestion")(
            "authority",
            "speed",
            "..."
        )();
        %
        leftToRight.top(30)(CTC, Wayside, TrackModel, TrainModel, TrainController);
        leftToRight(30)(Convert, BlockStatus, Suggestion);
        topToBottom(60)(Environment, Wayside);
        topToBottom(60)(SimRunner, TrainModel);
        topToBottom(60)(TrackModel, BlockStatus);
        % 
        drawObjects(CTC, Wayside, TrackModel, TrainModel, TrainController, Environment, SimRunner, Convert, BlockStatus, Suggestion);
        %
        % CTC, Wayside, TrackModel, TrainModel, TrainController, SimRunner ->  Environment
        link(associationUni)(CTC.n -- Environment.s);
        link(associationUni)(Wayside.n -- Environment.s);
        link(associationUni)(TrackModel.n -- Environment.s);
        link(associationUni)(TrainModel.n -- Environment.s);
        link(associationUni)(TrainController.n -- Environment.s);
        link(associationUni)(SimRunner.w -- Environment.e);
        % SimRunner -> CTC, Wayside, TrackModel, TrainModel, TrainController
        link(associationUni)(SimRunner.s -- CTC.n);
        link(associationUni)(SimRunner.s -- Wayside.n);
        link(associationUni)(SimRunner.s -- TrackModel.n);
        link(associationUni)(SimRunner.s -- TrainModel.n);
        link(associationUni)(SimRunner.s -- TrainController.n);
        % CTC <-> Wayside
        link(association)(CTC.e -- Wayside.w);
        % Wayside <-> TrackModel
        link(association)(Wayside.e -- TrackModel.w);
        % TrackModel <-> TrainModel
        link(association)(TrackModel.e -- TrainModel.w);
        % TrainModel <-> TrainController
        link(association)(TrainModel.e -- TrainController.w);
        % CTC, Wayside, TrackModel, TrainModel, TrainController ->  Convert
        link(associationUni)(CTC.s -- Convert.n);
        link(associationUni)(Wayside.s -- Convert.n);
        link(associationUni)(TrackModel.s -- Convert.n);
        link(associationUni)(TrainModel.s -- Convert.n);
        link(associationUni)(TrainController.s -- Convert.n);
        % CTC, Wayside ->  Suggestion
        link(associationUni)(CTC.s -- Suggestion.n);
        link(associationUni)(Wayside.s -- Suggestion.n);
        % CTC, Wayside, TrackModel ->  BlockStatus
        link(associationUni)(CTC.s -- BlockStatus.n);
        link(associationUni)(Wayside.s -- BlockStatus.n);
        link(associationUni)(TrackModel.s -- BlockStatus.n);
        % % StaticSwitch -> StaticBlock 1..3 
        % link(compositionUni)(pathStepX(StaticBlock.e, StaticSwitch.e, 30));
        % item(iAssoc)("1")(obj.nw = StaticSwitch.e + (15,0));
        % item(iAssoc)("3")(obj.nw = StaticBlock.e + (5,-5));
        % % TrackModel -> StaticSwitch 1..*
        % link(compositionUni)(pathStepX(StaticSwitch.e + (0,-20), TrackModel.e, 55));
        % item(iAssoc)("*")(obj.nw = StaticSwitch.e + (15,-20));
        % item(iAssoc)("1")(obj.nw = TrackModel.e + (5,-5));
        % % TrackModel -> StaticBlock 1..*
        % link(compositionUni)(pathStepX(StaticBlock.w, TrackModel.w, -25));
        % item(iAssoc)("*")(obj.nw = StaticBlock.w + (-20,0));
        % item(iAssoc)("1")(obj.nw = TrackModel.w + (-20,0));
        % % TrackModel -> Database 1..1
        % link(associationUni)(pathStepX(Database.n, TrackModel.s, 0));
        % item(iAssoc)("1")(obj.ne = TrackModel.s + (-5,-10));
        % item(iAssoc)("1")(obj.ne = Database.n + (0,10));
        % % StaticBlock -> StaticSwitch 1..0-1
        % link(associationUni)(pathStepX(StaticBlock.w + (0, -20), StaticSwitch.w, -25));
        % item(iAssoc)("1")(obj.nw = StaticBlock.w + (-20,-20));
        % item(iAssoc)("0..1")(obj.nw = StaticSwitch.w + (-20,-5));
	\end{mpost}
}
\end{center}

\section{Sequence Diagrams}

\subsection{Sequence Diagram: Initialize Simulation}
\begin{center}
\resizebox{\textwidth}{!}{
    \begin{sequencediagram}
    \newthread{sim}{SimRunner}
    \newinst{ctc}{CTC}
    \newinst{ws}{Wayside}
    \newinst{trackm}{TrackModel}
    \newinst{trainm}{TrainModel}
    \newinst{trainc}{TrainController}
    
    \begin{call}{sim}{init()}{trackm}{}
    \end{call}

    \begin{call}{sim}{init()}{ws}{}
    \end{call}

    \begin{call}{sim}{init()}{ctc}{}
    \end{call}

    \begin{call}{sim}{init()}{trainm}{}
    \end{call}

    \begin{call}{sim}{init()}{trainc}{}
    \end{call}

    \begin{sdblock}{Begin Simulation Loop}{}
        \begin{call}{sim}{...}{ctc}{}
        \end{call}
    \end{sdblock}
    \end{sequencediagram}
}
\end{center}

\subsection{Sequence Diagram: Run Simulation}
\begin{center}
\resizebox{!}{.9\textheight}{
    \begin{sequencediagram}
    \newthread{sim}{SimRunner}
    \newinst{ctc}{CTC}
    \newinst[3]{ws}{Wayside}
    \newinst[3]{trackm}{TrackModel}
    \newinst[3]{trainm}{TrainModel}
    \newinst[3]{trainc}{TrainController}
    
    \begin{call}{sim}{Increment Time}{sim}{}
    \end{call}

    \begin{call}{sim}{suggest()}{ctc}{}
        \begin{call}{ctc}{isOccupied(id)}{ws}{True/False}
            \begin{call}{ws}{isOccupied(id)}{trackm}{True/False}
                \begin{sdblock}{Update}{For Each Train}
                    \begin{call}{trackm}{getDisplacement(id)}{trainm}{displacement}
                        \begin{call}{trainm}{getPower()}{trainc}{power}
                            \begin{call}{trainc}{getSuggestedSpeed()}{trainm}{}
                                \begin{call}{trainm}{getSuggestedSpeed(id)}{trackm}{}
                                \end{call}
                            \end{call}
                            \begin{call}{trainc}{getSuggestedAuthority()}{trainm}{}
                                \begin{call}{trainm}{getSuggestedAuthority(id)}{trackm}{}
                                \end{call}
                            \end{call}
                            \begin{call}{trainc}{getBeacon()}{trainm}{}
                                \begin{call}{trainm}{getBeacon(id)}{trackm}{}
                                \end{call}
                            \end{call}
                            \begin{call}{trainc}{getBlockChange()}{trainm}{}
                                \begin{call}{trainm}{getBlockChange(id)}{trackm}{}
                                \end{call}
                            \end{call}
                        \end{call}
                    \end{call}
                \end{sdblock}
            \end{call}
        \end{call}
        \begin{sdblock}{Shallow Calls as Needed}{}
            \begin{call}{ctc}{getOccupancy(id), ...}{ws}{}
                \begin{call}{ws}{getOccupancy(id), ...()}{trackm}{}
                \end{call}
            \end{call}
        \end{sdblock}
        \begin{call}{ctc}{suggest(suggestions)}{ws}{}
            \begin{sdblock}{For Each Block}{}
                \begin{call}{ws}{setSpeed(...)}{trackm}{}
                \end{call}
                \begin{call}{ws}{setAuthority(...)}{trackm}{}
                \end{call}
                \begin{sdblock}{If a RR Crossing}{}
                    \begin{call}{ws}{setCrossingState(...)}{trackm}{}
                    \end{call}
                \end{sdblock}
            \end{sdblock}
            \begin{sdblock}{For Each Switch}{}
                \begin{call}{ws}{setSwitch(...)}{trackm}{}
                \end{call}
            \end{sdblock}
        \end{call}
    \end{call}
    \end{sequencediagram}
}
\end{center}

\subsection{Sequence Diagram: Dispatch Train}
\begin{center}
\resizebox{\textwidth}{!}{
 	\begin{sequencediagram}
	\newthread{disp}{Dispatcher}
	\newinst[1.8]{ctc}{CTCModel}
	%\newinst{ws}{Wayside}
	\newinst{tm}{TrackModel}
	\newinst[1.5]{train}{TrainModel}
	\newinst[2]{trainc}{TrainController}
	\newinst[1.5]{ui}{UI}
	
	\begin{call}{disp}{Dispatch Train}{ctc}{}
		\begin{call}{ctc}{createTrain(...)}{train}{}
			\begin{call}{train}{new TrainController()}{trainc}{}
				\begin{call}{trainc}{kp and ki popup}{ui}{}
				\end{call}
			\end{call}
			\begin{call}{train}{initializeTrain()}{tm}{}
			\end{call}
		\end{call}
	\end{call}
	\end{sequencediagram}
}
\end{center}

\subsection{Sequence Diagram: Dispatch Train (Variant: Invalid Inputs)}
\begin{center}
\resizebox{\textwidth}{!}{
 	\begin{sequencediagram}
	\newthread{disp}{Dispatcher}
	\newinst[1]{ctcgui}{CTCGUI}
	\newinst[8]{ctc}{CTCModel}
	%\newinst{ctcdat}{CTCTrainData}
	%\newinst{ws}{Wayside}
	%\newinst{tm}{TrackModel}
	%\newinst{train}{TrainModel}
	%\newinst{trainc}{TrainController}
	%\newinst{ui}{UI}
	
	\begin{call}{disp}{Dispatch Train}{ctcgui}{}
		\begin{call}{ctcgui}{checkTrainInputs(block, speed, authority, destination)}{ctc}{}
		\end{call}
	\end{call}
	
	\end{sequencediagram}
}
\end{center}

\subsection{Sequence Diagram: Upload PLC}
\begin{center}
\resizebox{\textwidth}{!}{
    \begin{sequencediagram}
        \newthread{weng}{Wayside Engineer}
        \newinst[3]{wc}{Wayside}
        \newinst[3]{comp}{Compiler}
        \newinst[3]{rf}{RegFile}
        \newinst[3]{pop}{Popup}
        
        \begin{call}
            {weng}{chooseFile()}{wc}{}
            \begin{call}
                {wc}{compile(file)}{comp}{Instruction[]}
            \end{call}
            \begin{call}
                {wc}{load(Instruction[])}{rf}{}
            \end{call}
            \begin{call}
                {wc}{success()}{pop}{}
            \end{call}
        \end{call}
    \end{sequencediagram}
}
\end{center}

\subsection{Sequence Diagram: Upload PLC (Variant: Unable to Read File)}
\begin{center}
\resizebox{\textwidth}{!}{
    \begin{sequencediagram}
        \newthread{weng}{Wayside Engineer}
        \newinst[3]{wc}{Wayside}
        \newinst[3]{comp}{Compiler}
        \newinst[3]{rf}{RegFile}
        \newinst[3]{pop}{Popup}
        
        \begin{call}
            {weng}{chooseFile()}{wc}{}
            \begin{call}
                {wc}{fileReadFailure()}{pop}{}
            \end{call}
        \end{call}
    \end{sequencediagram}
}
\end{center}

\subsection{Sequence Diagram: Upload PLC (Variant: Invalid PLC Code)}
\begin{center}
\resizebox{\textwidth}{!}{
    \begin{sequencediagram}
        \newthread{weng}{Wayside Engineer}
        \newinst[3]{wc}{Wayside}
        \newinst[3]{comp}{Compiler}
        \newinst[3]{rf}{RegFile}
        \newinst[3]{pop}{Popup}
        
        \begin{call}
            {weng}{chooseFile()}{wc}{}
            \begin{call}
                {wc}{compile(file)}{comp}{Instruction[]}
            \end{call}
            \begin{call}
                {wc}{compileError()}{pop}{}
            \end{call}
        \end{call}
    \end{sequencediagram}
}
\end{center}

\subsection{Sequence Diagram: Set Target Velocity}
\begin{center}
\resizebox{\textwidth}{!}{
 	\begin{sequencediagram}
	\newthread{dr}{Driver}
	\newinst[4]{ui}{UI}
	\newinst[4]{tc}{TrainController}
	
	\begin{call}{dr}{setVelocity()}{ui}{}
		\begin{call}{ui}{setVelocity()}{tc}{}
			\begin{call}{tc}{validate}{tc}{}
			\end{call}
		\end{call}
	\end{call}
	\end{sequencediagram}
}
\end{center}

\subsection{Sequence Diagram: Get Power}
\begin{center}
\resizebox{\textwidth}{!}{
 	\begin{sequencediagram}
	\newthread{tm}{TrainModel}
	\newinst[4]{tc}{TrainController}
	
	\begin{call}{tm}{getPower()}{tc}{double power}
		\begin{call}{tc}{getSuggestedSpeed()}{tm}{double speed}
		\end{call}
		\begin{call}{tc}{getMaxPower()}{tm}{double power}
		\end{call}
		\begin{call}{tc}{getVelocity()}{tm}{double speed}
		\end{call}
		\begin{call}{tc}{getAuthority()}{tm}{double authority}
		\end{call}
	\end{call}
	\end{sequencediagram}
}
\end{center}

\subsection{Sequence Diagram: Apply Service Brakes}
\begin{center}
\resizebox{\textwidth}{!}{
 	\begin{sequencediagram}
	\newthread{dr}{Driver}
	\newinst[4]{ui}{UI}
	\newinst[4]{tc}{TrainController}
	
	\begin{call}{dr}{applySBrakes()}{ui}{}
		\begin{call}{ui}{applySBrakes()}{tc}{}
			\begin{call}{tc}{validate}{tc}{}
			\end{call}
		\end{call}
	\end{call}
	\end{sequencediagram}
}
\end{center}

\subsection{Sequence Diagram: Apply Emergency Brakes}
\begin{center}
\resizebox{\textwidth}{!}{
 	\begin{sequencediagram}
	\newthread{dr}{Driver}
	\newinst[4]{ui}{UI}
	\newinst[4]{tc}{TrainController}
	
	\begin{call}{dr}{applyEBrakes()}{ui}{}
		\begin{call}{ui}{applyEBrakes()}{tc}{}
			\begin{call}{tc}{validate}{tc}{}
			\end{call}
		\end{call}
	\end{call}
	\end{sequencediagram}
}
\end{center}

\subsection{Sequence Diagram: Operate Lights}
\begin{center}
\resizebox{\textwidth}{!}{
 	\begin{sequencediagram}
	\newthread{dr}{Driver}
	\newinst[4]{ui}{UI}
	\newinst[4]{tc}{TrainController}
	
	\begin{call}{dr}{setLights(T/F)}{ui}{T/F}
		\begin{call}{ui}{setLights(T/F)}{tc}{T/F}
			\begin{call}{tc}{validate}{tc}{}
			\end{call}
		\end{call}
	\end{call}
	\end{sequencediagram}
}
\end{center}

\subsection{Sequence Diagram: Operate Doors}
\begin{center}
\resizebox{\textwidth}{!}{
 	\begin{sequencediagram}
	\newthread{dr}{Driver}
	\newinst[4]{ui}{UI}
	\newinst[4]{tc}{TrainController}
	
	\begin{call}{dr}{setDoors(T/F, T/F)}{ui}{byte status}
		\begin{call}{ui}{setDoors(T/F, T/F)}{tc}{byte status}
			\begin{call}{tc}{validate}{tc}{}
			\end{call}
		\end{call}
	\end{call}
	\end{sequencediagram}
}
\end{center}

\subsection{Sequence Diagram: Set Temperature}
\begin{center}
\resizebox{\textwidth}{!}{
 	\begin{sequencediagram}
	\newthread{dr}{Driver}
	\newinst[4]{ui}{UI}
	\newinst[4]{tc}{TrainController}
	
	\begin{call}{dr}{setTemp(temperature)}{ui}{double temperature}
		\begin{call}{ui}{setTemp(temperature)}{tc}{double temperature}
			\begin{call}{tc}{validate}{tc}{}
			\end{call}
		\end{call}
	\end{call}
	\end{sequencediagram}
}
\end{center}

\subsection{Sequence Diagram: Initialize Train Controller}
\begin{center}
\resizebox{\textwidth}{!}{
 	\begin{sequencediagram}
	\newthread{tm}{TrainModel or Simulation Runner}
	\newinst[4]{tc}{TrainController}
	
	\begin{call}{tm}{initialize}{tc}{Train Controller Instance}
		\begin{call}{tc}{setKi()}{tc}{}
		\end{call}
		\begin{call}{tc}{setKp()}{tc}{}
		\end{call}
	\end{call}
	\end{sequencediagram}
}
\end{center}



% THIS IS WHAT YOU WANT TO EDIT!
% ...if you want to include your own file in the full document.
% Include the name of your file - Without the ".tex"! - in a comment,
% and the contents of your file, after "\begin{document}", and 
% before "\end{document}" will be included.
% You want this comment to be on a line all alone,
% with no other text at all except the name of your file.
% YOU ALSO WANT TO MODIFY THE SCRIPT!
% Open up "scripts/splice.py" and follow the instructions there.
% ctc_model
% wayside
% track_model
% TrainModel
% train_controller

\end{document}