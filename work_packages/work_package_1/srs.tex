\documentclass{scrreprt}
\usepackage{listings}
\usepackage{underscore}
\usepackage[bookmarks=true]{hyperref}
\usepackage[utf8]{inputenc}
\usepackage[english]{babel}
\hypersetup{
    pdftitle={Software Requirement Specification},    % title
    pdfauthor={Training Montage},                     % author
    pdfsubject={TeX and LaTeX},                        % subject of the document
    pdfkeywords={TeX, LaTeX, graphics, images}, % list of keywords
    colorlinks=true,       % false: boxed links; true: colored links
    linkcolor=blue,       % color of internal links
    citecolor=black,       % color of links to bibliography
    filecolor=black,        % color of file links
    urlcolor=purple,        % color of external links
    linktoc=page            % only page is linked
}

\def\myversion{0.1 }
\date{}

\usepackage{hyperref}
\begin{document}

\begin{flushright}
    \rule{16cm}{5pt}\vskip1cm
    \begin{bfseries}
        \Huge{SOFTWARE REQUIREMENTS\\ SPECIFICATION}\\
        \vspace{.9cm}
        for\\
        \vspace{.9cm}
        COE 1186 Project\\
        \vspace{.9cm}
        \LARGE{Version \myversion approved}\\
        \vspace{.9cm}
        Prepared by:\\
        Alec Rosenbaum\\
        Aric Hudson\\
        Issac Goss\\
        Mitch Moran\\
        Parth Dadhania\\
        \vspace{1.9cm}
        Training Montage\\
        \vspace{.9cm}
        \today\\
    \end{bfseries}
\end{flushright}

\tableofcontents

\chapter*{Revision History}

\begin{center}
    \begin{tabular}{|c|c|c|c|}
        \hline
	    Name & Date & Reason For Changes & Version\\
        \hline
	    Alec Rosenbaum & September 14, 2017 & init & 0.1\\
        % \hline
	    % 31 & 32 & 33 & 34\\
        \hline
    \end{tabular}
\end{center}

\chapter{Introduction}

\section{Purpose}
The product whose software requirements are specified in this document is the
Training Montage Rail Simulator (TMRS). This SRS covers the functional and 
non-functional requirements, design contraints, and funcitonal limitations that
will be observed during the development of this project. This SRS will encompass
the entire system, and will include specific requirements, constraints, and 
limitations for each sub-system.

\section{Document Conventions}
The following acronyms will be used throughout the paper:
\begin{description}
  \item[TMRS] Training Montage Rail Simulator
  \item[Failure] A subsystem has failed when it allows two locomotives to share a block.
  \item[Unsafe] A situation or subsystem is unsafe if failure is made possible by said situation or subsystem.
  \item[Vital] A vital subsystem is one which shall not allow unsafe output.
  \item[PLC] Programmable Logic Controller, a hardware platform which operates solely on binary values, controlled by code given by a user. The term PLC will be used interchangeably to refer to the device and the code which runs on it.
  \item[Block] The track is partitioned into blocks, what is considered an indivisible unit of track. A block has one speed limit, one grade, one altitude, no switches, and at most one train occupying it.
\end{description}

$<$Describe any standards or typographical conventions that were followed when 
writing this SRS, such as fonts or highlighting that have special significance.  
For example, state whether priorities  for higher-level requirements are assumed 
to be inherited by detailed requirements, or whether every requirement statement 
is to have its own priority.$>$

\section{Intended Audience and Reading Suggestions}
This document is intended as a part of the contractual agreement between Training
Montage and their client for the TMRS software package. It will be used as reference
by developers, project managers, testers, and the client.

It is suggested that the user read over the project scope and perspective, then
visit the UI design prototype included in the Appendix. The UI design prototype 
will give a good sense of how users will interact with the software, and of what
capabilities are provided by the system. After reviewing the UI design prototype,
proceed to read the rest of the document in the order that it's presented.

\section{Project Scope}
$<$Provide a short description of the software being specified and its purpose, 
including relevant benefits, objectives, and goals. Relate the software to 
corporate goals or business strategies. If a separate vision and scope document 
is available, refer to it rather than duplicating its contents here.$>$

\section{References}
$<$List any other documents or Web addresses to which this SRS refers. These may 
include user interface style guides, contracts, standards, system requirements 
specifications, use case documents, or a vision and scope document. Provide 
enough information so that the reader could access a copy of each reference, 
including title, author, version number, date, and source or location.$>$


\chapter{Overall Description}

\section{Product Perspective}
$<$Describe the context and origin of the product being specified in this SRS.  
For example, state whether this product is a follow-on member of a product 
family, a replacement for certain existing systems, or a new, self-contained 
product. If the SRS defines a component of a larger system, relate the 
requirements of the larger system to the functionality of this software and 
identify interfaces between the two. A simple diagram that shows the major 
components of the overall system, subsystem interconnections, and external 
interfaces can be helpful.$>$

\section{Product Functions}
$<$Summarize the major functions the product must perform or must let the user 
perform. Details will be provided in Section 3, so only a high level summary 
(such as a bullet list) is needed here. Organize the functions to make them 
understandable to any reader of the SRS. A picture of the major groups of 
related requirements and how they relate, such as a top level data flow diagram 
or object class diagram, is often effective.$>$

\section{User Classes and Characteristics}
This produce is expected to be used by the following classes of users:
\begin{enumerate}
    \item Dispatcher
    \item Wayside Engineer
    \item Train Conductor
\end{enumerate}

It is expected none of these user classes to have technical expertise. As such,
each user will be provided with an graphical user interface that allows full
functionality of each module.

It is, however, expected that Wayside Engineers are experts both on a section
of track and in writing PLC code. All uploaded code is considered vital.

\section{Operating Environment}
The software will operate on the Java Virtual Machine. The Java Virtual Machine
can be installed on computers with a Pentium 2 266 MHz or faster processor, at
least 128 MB of physical RAM, and 124MB of free disk space. Java supports Windows,
Mac OS X, Linux, and Solaris. We will be writing software using Java Version 8.

\section{Design and Implementation Constraints}
$<$Describe any items or issues that will limit the options available to the 
developers. These might include: corporate or regulatory policies; hardware 
limitations (timing requirements, memory requirements); interfaces to other 
applications; specific technologies, tools, and databases to be used; parallel 
operations; language requirements; communications protocols; security 
considerations; design conventions or programming standards (for example, if the 
customer’s organization will be responsible for maintaining the delivered 
software).$>$

\section{User Documentation}

The following documentation components will be delivered along with the software:
\begin{enumerate}
    \item UI Design
    \item User's Manual
\end{enumerate}

\section{Assumptions and Dependencies}

$<$List any assumed factors (as opposed to known facts) that could affect the 
requirements stated in the SRS. These could include third-party or commercial 
components that you plan to use, issues around the development or operating 
environment, or constraints. The project could be affected if these assumptions 
are incorrect, are not shared, or change. Also identify any dependencies the 
project has on external factors, such as software components that you intend to 
reuse from another project, unless they are already documented elsewhere (for 
example, in the vision and scope document or the project plan).$>$


\chapter{External Interface Requirements}

\section{User Interfaces}
There will a be discrete user interface for each System Module. These user
interfaces will be capable of running seperately, but may also be run together
in order to show the full functionality of each module and how it interfaces
behind-the-scenes with other modules.

UI Prototypes for each module are shown in the UI Design document. The User's
Manual further elaborates on operational details and how users are expected to
interface with each module.

$<$Describe the logical characteristics of each interface between the software 
product and the users. This may include sample screen images, any GUI standards 
or product family style guides that are to be followed, screen layout 
constraints, standard buttons and functions (e.g., help) that will appear on 
every screen, keyboard shortcuts, error message display standards, and so on.  
Define the software components for which a user interface is needed. Details of 
the user interface design should be documented in a separate user interface 
specification.$>$

\section{Hardware Interfaces}
$<$Describe the logical and physical characteristics of each interface between 
the software product and the hardware components of the system. This may include 
the supported device types, the nature of the data and control interactions 
between the software and the hardware, and communication protocols to be 
used.$>$

\section{Software Interfaces}
$<$Describe the connections between this product and other specific software 
components (name and version), including databases, operating systems, tools, 
libraries, and integrated commercial components. Identify the data items or 
messages coming into the system and going out and describe the purpose of each.  
Describe the services needed and the nature of communications. Refer to 
documents that describe detailed application programming interface protocols.  
Identify data that will be shared across software components. If the data 
sharing mechanism must be implemented in a specific way (for example, use of a 
global data area in a multitasking operating system), specify this as an 
implementation constraint.$>$

\section{Communications Interfaces}
$<$Describe the requirements associated with any communications functions 
required by this product, including e-mail, web browser, network server 
communications protocols, electronic forms, and so on. Define any pertinent 
message formatting. Identify any communication standards that will be used, such 
as FTP or HTTP. Specify any communication security or encryption issues, data 
transfer rates, and synchronization mechanisms.$>$


\chapter{System Features}
$<$This template illustrates organizing the functional requirements for the 
product by system features, the major services provided by the product. You may 
prefer to organize this section by use case, mode of operation, user class, 
object class, functional hierarchy, or combinations of these, whatever makes the 
most logical sense for your product.$>$

\section{Train Controller}

\subsection{Description}
This module is responsible for safely dictating the train's speed by way of controlling its power, as well as controlling some smaller tasks like light and door control, and feedback about current and approaching stations. This module is VITAL.

This vital controller receives a suggested speed and compares it to its own calculated maximum safe speed and selects the lower of the two values to relay to the Train Model. As the train's speed is controlled by its current power, the Train Controller converts the new speed into a power requirement (or a brake setting, if needed).  The train controller also dictates when the lights and doors should be activated, and announces when the train is approaching a station or has stopped at a station.

\subsection{I/O}
The Train Controller receives a suggested speed from the Wayside Controller. This information will be processed and relayed to the Train Model as a power setting.  The controller sends signals to turn lights on or off, and open doors on either side of the train.  This module receives feedback from the Train Model in the from of its current speed and/or power.  The Train Controller also knows the train's current authority in the form of distance in miles.

\subsection{Interface}
\begin{enumerate}
\item The Train Controller must have an algorithm capable of calculating a maximum safe speed for the train in question.
	\begin{enumerate}
		\item The maximum safe speed will be the maximum speed that can be fully arrested in the minimum safe breaking distance.
		\item The train controller shall consider the train's mass, current speed, and current authority when calculating this speed.
		\item Details of this algorithm are TBD.
	\end{enumerate}

\item The Train Controller shall select a safe speed for the train.
	\begin{enumerate}
		\item The Train Controller shall choose the suggested pseed or its own calculated maximum safe speed.
		\item By default, the lower of the two speeds will be chosen for the train.
		\item The Train Controller shall allow a manual mode for the driver of the train to manually input a speed.  This speed shall not exceed the speed suggested by the Wayside Controller.
	\end{enumerate}

\item The module, once it has chosen a speed for the train, will require an algorithm to convert that desired speed into either (1) a power setting for the train, or (2) an application of the brake for a certain amount of time.
	\begin{enumerate}
		\item This algorithm shall use the train's current and desired kinetic energy combined with a time constraint to determine the power required to achieve the desired speed.
		\item This algorithm shall also take grade into account.
		\item The details of this algorithm are TBD.
	\end{enumerate}

\item The Train Controller must dictate that lights should be turned on or off.
	\begin{enumerate}
		\item The Train Controller shall control lights based on a daily schedule.
		\item It is the responsibility of the Train Model to execute these instructions.
	\end{enumerate}

\item The Train Controller must dictate that train doors should open.
	\begin{enumerate}
		\item The Train Controller shall only open doors when the train is stopped.
		\item Doors can be opened on the left or the right, or both.
		\item The Train Controller shall read which doors should be open from beacons approaching each station.
		\item In the event of an emergency stop and train evacuation, the Train Controller shall send the command to open whichever doors are appropriate, given the situation at hand.
		\item It is the responsibility of the Train Model to execute these instructions.
	\end{enumerate}

\item The Train Controller must display stops and stations.
	\begin{enumerate}
		\item The upcoming stop and the distance to that stop shall be read from an RFID sensor one block before the station or stop's location.
		\item The Train Controller shall display the stop as soon as it receives and deciphers the information from the RFID sensor.
		\item The Train Controller shall display the information until the train has physically left that station.
	\end{enumerate}
\end{enumerate}

\chapter{System Modules}
This application shall be made of 5 separate modules, each of which a team member shall own.
These shall be the Centralized Traffic Controller, Wayside / Track Controller, Track Model, Train Model, and Train Controller.
These modules shall have strict communication constraints and be organized in the following way.
% an image will be imput here

\section{Track Model}

\subsection{Description}
The system shall have a model of the transit system track layout. This module is not VITAL

\subsection{Interface}
\begin{enumerate}
    \item The track model shall accept speed and authority as inputs from the Wayside Controller.
    \item The track model shall provide block occupancy and switch positions to the Wayside Controller.
    \item The track model shall provide speed and authority to the Train Model.
\end{enumerate}

\subsection{Functional Requirements}
\begin{enumerate}
    \item The Track Model shall consider grade and elevation.
    \item The Track Model shall be configurable.
    \item The Track Model shall consider allowable directions of travel, branching, and speed limits.
    \item The Track Model shall be able to export and import track layouts.
    \item The Track Model shall consider block size.
    \begin{enumerate}
        \item Blocks must be shown and configurable.
    \end{enumerate}
    \item The Track Model shall signals and switch machines.
    \item The Track Model shall implement track circuits for presence detection.
    \item The Track Model shall consider railway crossings.
    \item The Track Model shall include stations.
    \begin{enumerate}
        \item Passengers shall be loaded and unloaded at stations.
    \end{enumerate}
    \item The Track Model shall implement the following failure modes:
    \begin{enumerate}
        \item Broken rail
        \item Track Circuit failure
        \item Extra or no trains detected
        \item Power failure
        \item No communication going to train
    \end{enumerate}
\end{enumerate}

\section{Wayside / Track Controller}

\subsection{Description}
This module is a vital computing platform.
It shall output safe (not unsafe) speed or authority to the Track Model.
It shall also control track flow, based on PLC code given by a Wayside Engineer.

\begin{enumerate}
    \item The Wayside Controller shall accept suggested (non-vital) speed and authority (per train) from the CTC.
    \item The Wayside Controller shall accept block occupancy and switch positions from the Track Model.
    \item The Wayside Controller shall have vital outputs to the Track Model:
        \begin{enumerate}
            \item Speed \& Authority per train.
            \item Light Colors per track block.
            \item Switch Positions.
            \end{enumerate}
\end{enumerate}

\chapter{Other Nonfunctional Requirements}

\section{Performance Requirements}
The simulation software must be able to simulate at 10 times wall clock speed.
This will allow quicker evaluation of a simulated circumstances. The simulation
speed should be modifyable during simulation; i.e. it specified while the simulation
is running, rather than before the software starts.

\section{Safety Requirements}
No train shall at any point crash into any other train. No train shall exceed
the speed limit or authority.
$<$Specify those requirements that are concerned with possible loss, damage, or 
harm that could result from the use of the product. Define any safeguards or 
actions that must be taken, as well as actions that must be prevented. Refer to 
any external policies or regulations that state safety issues that affect the 
product’s design or use. Define any safety certifications that must be 
satisfied.$>$

\section{Security Requirements}
$<$Specify any requirements regarding security or privacy issues surrounding use 
of the product or protection of the data used or created by the product. Define 
any user identity authentication requirements. Refer to any external policies or 
regulations containing security issues that affect the product. Define any 
security or privacy certifications that must be satisfied.$>$

\section{Software Quality Attributes}
$<$Specify any additional quality characteristics for the product that will be 
important to either the customers or the developers. Some to consider are: 
adaptability, availability, correctness, flexibility, interoperability, 
maintainability, portability, reliability, reusability, robustness, testability, 
and usability. Write these to be specific, quantitative, and verifiable when 
possible. At the least, clarify the relative preferences for various attributes, 
such as ease of use over ease of learning.$>$

\section{Business Rules}
$<$List any operating principles about the product, such as which individuals or 
roles can perform which functions under specific circumstances. These are not 
functional requirements in themselves, but they may imply certain functional 
requirements to enforce the rules.$>$


\chapter{Other Requirements}
$<$Define any other requirements not covered elsewhere in the SRS. This might 
include database requirements, internationalization requirements, legal 
requirements, reuse objectives for the project, and so on. Add any new sections 
that are pertinent to the project.$>$

\section{Appendix A: Glossary}
%see https://en.wikibooks.org/wiki/LaTeX/Glossary
$<$Define all the terms necessary to properly interpret the SRS, including 
acronyms and abbreviations. You may wish to build a separate glossary that spans 
multiple projects or the entire organization, and just include terms specific to 
a single project in each SRS.$>$

\section{Appendix B: Analysis Models}
$<$Optionally, include any pertinent analysis models, such as data flow 
diagrams, class diagrams, state-transition diagrams, or entity-relationship 
diagrams.$>$

\section{Appendix C: To Be Determined List}
$<$Collect a numbered list of the TBD (to be determined) references that remain 
in the SRS so they can be tracked to closure.$>$

\end{document}
